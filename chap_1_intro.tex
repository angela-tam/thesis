% Chapter 1

\chapter{Introduction} % Main chapter title

\label{Chapter1} % Change X to a consecutive number; for referencing this chapter elsewhere, use \ref{ChapterX}

%----------------------------------------------------------------------------------------
%	SECTION 1
%----------------------------------------------------------------------------------------
\section{General context}

Alzheimer's disease (AD) is a progressive neurodegenerative disorder and the most common cause of dementia. As AD is an age-related disorder, more and more individuals will develop AD as the population ages. For instance, there are currently 53 million adults aged 65 and older in the United States in 2018. This segment of the American population is projected to increase to 88 million by 2050, followed by an expected doubling of the number of AD cases \citep{Association:2018eq}. Under current circumstances for diagnosis and treatment, the cost of medical and long-term care expenditures for American individuals who will develop AD in 2018 is projected to be \$47.1 trillion USD \citep{Association:2018eq}. AD clearly presents a public health crisis that requires imminent solutions.
% maybe change to Canadian??

Unfortunately, there is currently no disease-modifying drug for AD that can reverse or slow down the course of the disease. The failures for clinical trials of disease-modifying agents have been blamed on lack of efficacy, among other reasons \citep{Cummings:2014gn}. The lack of efficacy could be due to drugs being tested too late in the disease process when irreversible neurodegeneration has already occurred in clinical trial participants \citep{Aisen:2013fj}. The pathology of AD develops over decades prior to the emergence of clinical symptoms. Individuals who develop AD dementia typically experience an intermediate prodromal phase called mild cognitive impairment (MCI), where some cognitive deficits are apparent but not severe enough to impede daily function. There is therefore a push to target pre-symptomatic individuals with disease-modifying agents to try to prevent disease progression. A second reason for a lack of demonstrated efficacy could be heterogeneity within clinical populations. AD exists in several forms, and an individual's risk for developing AD depends on a variety of factors including, but not limited to, genetics, cardiovascular health, and brain reserve. This heterogeneity may explain why inclusion criteria for clinical trials have had low to moderate positive predictive value for diagnosing MCI subjects who would develop AD dementia \citep{Visser:2005iw}. The failure of clinical trials at the MCI stage in order to prevent dementia may be partially attributed to the incorrect inclusion of individuals who will not develop AD dementia. The development of a biomarker for early and precise diagnosis that would account for this heterogeneity could greatly improve patient selection for clinical trials and could identify high-risk individuals for earlier interventions. 

% An early biomarker would also be potentially valuable for clinical use. An early diagnosis prior to the emergence of symptoms may allow for early interventions. Although there are currently no pharmacological treatments that can slow down AD-related neurodegeneration, there are therapies (both pharmacologial and non-pharmacological) that may decrease the rate of cognitive decline in patients. Aside from medical benefits, an early diagnosis may have an important financial impact, where AD-related health costs could be significantly reduced with earlier detection \citep{Association:2018eq}.


%----------------------------------------------------------------------------------------
%	SECTION 2
%----------------------------------------------------------------------------------------

\section{Objectives}

A promising tool that is already widely used in clinical settings to aid in the diagnosis of AD is magnetic resonance imaging (MRI). The overall objective of this thesis was to explore heterogeneity within MRI-based biomarkers in order to discover a brain signature that is highly predictive of Alzheimer's dementia. Such a signature could include, for example, atrophy in the medial temporal lobes and specific nodes of the default mode network, dysconnectivity within those regions, and cognitive deficits. We focused on two MRI modalities, structural and functional MRI, and examined their potential to explain cognitive outcomes. Specific objectives from each chapter that presents original research are described below.

\subsection*{Chapter 3 objectives}
Chapter 3 of this thesis assessed the robustness of resting-state connectivity derived from functional MRI as an early biomarker for AD, in the face of heterogeneity from different image acquisition and diagnostic protocols. To this end, we combined multiple independent datasets of resting-state functional magnetic resonance images from MCI and cognitively normal subjects to test the consistency of functional connectivity differences between these two diagnostic groups.  

\subsection*{Chapter 4 objectives}
Chapter 4 was a companion paper to Chapter 3 and described resting-state network parcellations that are present in a large heterogeneous population of older adults with or without MCI. 

\subsection*{Chapter 5 objectives}
Chapter 5 of this thesis characterized the variability in brain atrophy patterns, derived from structural MRI, and characterized the heterogeneity to develop a signature that has high positive predictive value at predicting incipient AD dementia in MCI subjects. We achieved this by identifying subtypes of brain atrophy in AD and control subjects. We then applied a machine learning algorithm to classify AD and controls based on brain atrophy subtypes and cognitive test scores. This resulted in a signature that was common to AD patients but not represented in controls. We then transfered the predictive model to identify a subset of individuals who carried the signature and who would progress to AD dementia from a dataset of MCI subjects.



