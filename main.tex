%%%%%%%%%%%%%%%%%%%%%%%%%%%%%%%%%%%%%%%%%
% Thesis LaTeX template by Angela Tam
% This template (mostly) conforms to the thesis requirements of McGill University
%
%%%%%%%%%%%%%%%%%%%%%%%%%%%%%%%%%%%%%%%%
% Please note that this work is derived from the following:
%
% Masters/Doctoral Thesis
% LaTeX Template
% Version 2.5 (27/8/17)
%
% This template was downloaded from:
% http://www.LaTeXTemplates.com
%
% Version 2.x major modifications by:
% Vel (vel@latextemplates.com)
%
% This template is based on a template by:
% Steve Gunn (http://users.ecs.soton.ac.uk/srg/softwaretools/document/templates/)
% Sunil Patel (http://www.sunilpatel.co.uk/thesis-template/)
%
% Template license:
% CC BY-NC-SA 3.0 (http://creativecommons.org/licenses/by-nc-sa/3.0/)
%
%%%%%%%%%%%%%%%%%%%%%%%%%%%%%%%%%%%%%%%%%

%----------------------------------------------------------------------------------------
%	PACKAGES AND OTHER DOCUMENT CONFIGURATIONS
%----------------------------------------------------------------------------------------

\documentclass[
12pt, % The default document font size, options: 10pt, 11pt, 12pt
oneside, % Two side (alternating margins) for binding by default, uncomment to switch to one side
english, % ngerman for German
onehalfspacing, % line spacing, options: singlespacing, onehalfspacing or doublespacing
%draft, % Uncomment to enable draft mode (no pictures, no links, overfull hboxes indicated)
%nolistspacing, % If the document is onehalfspacing or doublespacing, uncomment this to set spacing in lists to single
liststotoc, % Uncomment to add the list of figures/tables/etc to the table of contents
toctotoc, % Uncomment to add the main table of contents to the table of contents
%parskip, % Uncomment to add space between paragraphs
%nohyperref, % Uncomment to not load the hyperref package
headsepline, % Uncomment to get a line under the header
%chapterinoneline, % Uncomment to place the chapter title next to the number on one line
consistentlayout, % Uncomment to change the layout of the declaration, abstract and acknowledgements pages to match the default layout
]{McGillThesis} % The class file specifying the document structure

\usepackage[utf8]{inputenc} % Required for inputting international characters
\usepackage[T1]{fontenc} % Output font encoding for international characters

\usepackage{mathpazo} % Use the Palatino font by default

\usepackage[backend=bibtex,style=authoryear,natbib=true,dashed=false]{biblatex} % Use the bibtex backend with the authoryear citation style (which resembles APA)

\addbibresource{papers_library.bib} % The filename of the bibliography

\usepackage[autostyle=true]{csquotes} % Required to generate language-dependent quotes in the bibliography

\usepackage[section]{placeins} % force floats to stay within section

%\usepackage{floatrow} % library for side captions? (and probably other things)
\usepackage{sidecap} % library for side captions
\sidecaptionvpos{figure}{t} % force side captions to align to top of figure

% other options for captions
\captionsetup[table]{font=scriptsize,labelfont=scriptsize}
\captionsetup[figure]{font=scriptsize,labelfont=scriptsize}

\usepackage{pdfpages} % insert pdf files into latex doc

\usepackage[titletoc]{appendix} % appendices package

% allow for strike-out and underline
\usepackage[normalem]{ulem}

\usepackage[raggedright]{titlesec} % don't hyphenate titles/headings

% this sets the deepest level at which to number sections and subsections
\setcounter{tocdepth}{3}
\setcounter{secnumdepth}{3}

% package to make typeset look prettier
\usepackage{microtype}

% command to set font for headers
\renewcommand{\headfont}{\textsc} %\normalfont
%----------------------------------------------------------------------------------------
%	MARGIN SETTINGS
%----------------------------------------------------------------------------------------

\geometry{
	paper=a4paper, % Change to letterpaper for US letter
	inner=2.5cm, % Inner margin
	outer=2.5cm, % Outer margin
	bindingoffset=.5cm, % Binding offset
	top=1.5cm, % Top margin
	bottom=1.5cm, % Bottom margin
	%showframe, % Uncomment to show how the type block is set on the page
}

%----------------------------------------------------------------------------------------
%	THESIS INFORMATION
%----------------------------------------------------------------------------------------

\thesistitle{Predicting Alzheimer's dementia from heterogeneous patterns of neurodegeneration and functional connectivity} % Your thesis title, this is used in the title and abstract, print it elsewhere with \ttitle
\supervisor{Dr. James \textsc{Smith}} % Your supervisor's name, this is used in the title page, print it elsewhere with \supname
\examiner{} % Your examiner's name, this is not currently used anywhere in the template, print it elsewhere with \examname
\degree{Doctor of Philosophy} % Your degree name, this is used in the title page and abstract, print it elsewhere with \degreename
\author{\textsc{Angela Tam}} % Your name, this is used in the title page and abstract, print it elsewhere with \authorname
\addresses{} % Your address, this is not currently used anywhere in the template, print it elsewhere with \addressname

\subject{Neuroscience} % Your subject area, this is not currently used anywhere in the template, print it elsewhere with \subjectname
\keywords{} % Keywords for your thesis, this is not currently used anywhere in the template, print it elsewhere with \keywordnames
\university{\href{http://www.university.com}{University Name}} % Your university's name and URL, this is used in the title page and abstract, print it elsewhere with \univname
\department{\href{http://department.university.com}{Department or School Name}} % Your department's name and URL, this is used in the title page and abstract, print it elsewhere with \deptname
\group{\href{http://researchgroup.university.com}{Research Group Name}} % Your research group's name and URL, this is used in the title page, print it elsewhere with \groupname
\faculty{\href{http://faculty.university.com}{Faculty Name}} % Your faculty's name and URL, this is used in the title page and abstract, print it elsewhere with \facname

\AtBeginDocument{
\hypersetup{pdftitle=\ttitle} % Set the PDF's title to your title
\hypersetup{pdfauthor=\authorname} % Set the PDF's author to your name
\hypersetup{pdfkeywords=\keywordnames} % Set the PDF's keywords to your keywords
\hypersetup{citecolor=blue} % link colours
\hypersetup{linkcolor=blue}
\hypersetup{urlcolor=blue}
}

\begin{document}

\frontmatter % Use roman page numbering style (i, ii, iii, iv...) for the pre-content pages

\pagestyle{plain} % Default to the plain heading style until the thesis style is called for the body content

%----------------------------------------------------------------------------------------
%	TITLE PAGE
%----------------------------------------------------------------------------------------

\begin{titlepage}
\begin{center}

\vspace*{.06\textheight}
%{\scshape\LARGE \univname\par}\vspace{1.5cm} % University name
%\textsc{\Large Doctoral Thesis}\\[0.5cm] % Thesis type

\HRule \\[0.4cm] % Horizontal line
{\huge \bfseries \ttitle\par}\vspace{0.4cm} % Thesis title
\HRule \\[1.5cm] % Horizontal line


{\Large \authorname}\\ % Author name - remove the \href bracket to remove the link
Integrated Program in Neuroscience\\
McGill University, Montreal\\
August 2018

\vfill

\small \textit{A thesis submitted to McGill University in partial fulfillment\\ of the requirements of the degree of \degreename}\\[0.3cm] % University requirement text


\vfill

{\small \copyright\ Angela Tam, 2018}\\[1cm] % Date
%\includegraphics{Logo} % University/department logo - uncomment to place it

\vfill
\end{center}
\end{titlepage}

%----------------------------------------------------------------------------------------
%	ABSTRACT PAGE
%----------------------------------------------------------------------------------------

\begin{abstract}
\addchaptertocentry{\abstractname} % Add the abstract to the table of contents
There is a large field of research dedicated to the development of biomarkers for an early diagnosis of Alzheimer's disease (AD). Predicting AD dementia within an individual, especially at a prodromal stage like mild cognitive impairment (MCI), is complicated by the vast amount of heterogeneity present in populations. This thesis explores heterogeneity in brain organization with magnetic resonance imaging (MRI) in order to develop biomarkers to identify individuals who will progress to AD dementia. Chapter 1 provides a brief introduction to the problem at hand and lists the specific aims of this thesis. Chapter 2 provides a review of the literature of biomarker development for AD, with a focus on MRI-based studies and prediction of cognitive trajectories with machine learning. In chapter 3, we present a study that explored whether there were functional connections in resting-state networks that could consistently discriminate between patients with MCI and cognitively normal older adults in the face of heterogeneity from methodological procedures. We identified functional connections that were robustly altered in the default mode network and the cortical-strial-thalamic loop, albeit with small to medium effect sizes, in MCI patients compared to controls in several independent datasets. We also provide sample size estimates to obtain adequate statistical power in a multisite study setting. Chapter 4 presents a study describing resting-state networks at various spatial resolutions in a heterogeneous sample of older adults with and without cognitive impairment. In chapter 5, we describe a study in which we developed a signature based on brain atrophy patterns and cognitive deficits that is highly predictive of future progression to AD dementia in a subgroup of individuals with MCI. By harnessing the heterogeneity in brain structure, we were able to achieve higher positive predictive values and specificity compared to previous works at predicting progression to dementia from the MCI stage. Lastly, a discussion of the contributions and future developments from these studies is presented in chapter 6. This thesis provides novel insights into the heterogeneity of structural and functional brain organization and the use of MRI as a tool to develop biomarkers for AD.
\end{abstract}

%----------------------------------------------------------------------------------------
%	FRENCH ABSTRACT PAGE
%----------------------------------------------------------------------------------------

\begin{resume}
\addchaptertocentry{\resumename} % Add the abstract to the table of contents
Il y a un grand domaine de recherche dédié à l’identification précoce de la maladie d’Alzheimer (MA). La prédiction de la démence liée à la MA chez un individu, surtout dans ceux qui sont atteints des troubles cognitifs légers (TCL), est compliquée à cause de la variabilité au niveau des populations. Cette thèse explore l’hétérogénéité dans l’organisation du cerveau avec l’imagerie par résonance magnétique (IRM) pour l’objectif de développer des biomarqueurs capables d’identifier des individus qui seront atteints de la démence liée à la MA. Chapitre 1 introduit le problème et les objectifs spécifiques de la thèse. Chapitre 2 consiste en un survol de la littérature du développement des biomarqueurs pour la MA, avec un emphase sur les études d'IRM et la prédiction des trajets cognitifs avec l’apprentissage machine. Chapitre 3 offre un étude où on examine la présence des connections fonctionnels dans les réseaux d’état de repos qui pourraient discriminer systématiquement entre les patients avec des TCL et les personnes âgées cognitivement sains, malgré l’hétérogénéité des procédures méthodologiques. On a identifié des connections fonctionnels dans le réseau mode par défaut et le circuit entre le cortex, striatum et thalamus qui étaient altérés de manière robuste dans les patients avec des TCL comparés aux contrôles dans plusieurs jeux de données indépendants. Nous offrons aussi des estimations des tailles d’échantillon pour obtenir le pouvoir statistique adéquat dans un contexte de recherche multisite. Chapitre 4 présent un étude qui décrit les réseaux d’état de repos à travers de nombreux résolutions spatiaux dans un échantillon hétérogène de personnes âgées avec et sans des troubles cognitifs. Dans chapitre 5, on a développer une signature basée sur les patrons d’atrophie du cerveau et déficits cognitifs qui est hautement prédictive de la démence liée à la MA dans un sous-groupe d’individus avec TCL. En employant l’hétérogénéité de la structure du cerveau, on rapport des valeurs prédictive positives et des spécificités plus hausses que les études antérieures qui ont visé à prédire la progression vers la démence de l’étage des TCL. Finalement, le chapitre 6 porte sur une discussion des contributions et développements futurs de ces études. Cette thèse apporte des nouveaux apprentissages par rapport à l’hétérogénéité de l’organisation structurelle et fonctionnelle du cerveau et l’utilité de l’IRM comme un outil pour le développement des biomarqueurs de la MA.
\end{resume}



%----------------------------------------------------------------------------------------
%	LIST OF CONTENTS/FIGURES/TABLES PAGES
%----------------------------------------------------------------------------------------
\microtypesetup{protrusion=false}

\tableofcontents % Prints the main table of contents

\listoffigures % Prints the list of figures

\listoftables % Prints the list of tables

\microtypesetup{protrusion=true}
%----------------------------------------------------------------------------------------
%	ABBREVIATIONS
%----------------------------------------------------------------------------------------

\begin{abbreviations}{rl} % Include a list of abbreviations (a table of two columns)

\textbf{A$\beta$} & \textbf{A}myloid \textbf{$\beta$}\\
\textbf{AD} & \textbf{A}lzheimer's \textbf{D}isease\\
\textbf{ADNI} & \textbf{A}lzheimer's \textbf{D}isease \textbf{N}euroimaging \textbf{I}nitiative\\
\textbf{aMCI} & \textbf{a}mnestic \textbf{M}ild \textbf{C}ognitive \textbf{I}mpairment\\
\textbf{APOE} & \textbf{APO}lipoprotein \textbf{E}\\
\textbf{APP} & \textbf{A}myloid \textbf{P}recursor \textbf{P}rotein\\
\textbf{BASC} & \textbf{B}ootstrap \textbf{A}nalysis of \textbf{S}table \textbf{C}lusters\\
\textbf{CN} & \textbf{C}ognitively \textbf{N}ormal\\
\textbf{DMN} & \textbf{D}efault \textbf{M}ode \textbf{N}etwork\\
\textbf{fMRI} & \textbf{f}unctional \textbf{M}agnetic \textbf{R}esonance \textbf{I}maging\\
\textbf{HPS} & \textbf{H}ighly \textbf{P}redictive \textbf{S}ignature\\
\textbf{LDA} & \textbf{L}inear \textbf{D}iscriminant \textbf{A}nalysis\\
\textbf{MCI} & \textbf{M}ild \textbf{C}ognitive \textbf{I}mpairment\\
\textbf{MRI} & \textbf{M}agnetic \textbf{R}esonance \textbf{I}maging\\
\textbf{MSTEPS} & \textbf{M}ultiscale \textbf{STEP}wise \textbf{S}election\\
\textbf{PET} & \textbf{P}ositron \textbf{E}mission \textbf{T}omography\\
\textbf{pMCI} & \textbf{p}rogressive \textbf{M}ild \textbf{C}ognitive \textbf{I}mpairment\\
\textbf{PSEN1} & \textbf{P}re\textbf{SEN}ilin \textbf{1}\\
\textbf{PSEN2} & \textbf{P}re\textbf{SEN}ilin \textbf{2}\\
\textbf{sMCI} & \textbf{s}table \textbf{M}ild \textbf{C}ognitive \textbf{I}mpairment\\
\textbf{SVM} & \textbf{S}upport \textbf{V}ector \textbf{M}achine\\
\textbf{VBM} & \textbf{V}oxel-\textbf{B}ased \textbf{M}orphometry\\


\end{abbreviations}


%----------------------------------------------------------------------------------------
%	ACKNOWLEDGEMENTS
%----------------------------------------------------------------------------------------

\begin{acknowledgements}
\addchaptertocentry{\acknowledgementname} % Add the acknowledgements to the table of contents
I would first like to thank my supervisor Dr. Pierre Bellec for his support. I am without a doubt a significantly more competent scientist than when I first started my Ph.D. and I could not have done that without his investment in me. I cannot have imagined a better mentor and I will always be grateful for the lessons and opportunities that Pierre provided. I would like to thank my co-supervisor Dr. John Breitner for his support, guidance, and advice throughout my degree. I thank my SIMEXP labmates (past and present), Dr. Aman Badhwar, Sebastian Urchs, Yassine Benhajali, Perrine Ferré, Clara Moreau, Amal Boukdhir, Jonathan Armoza, Dr. Pierre-Olivier Quirion, Phil Dickinson, Dr. Yu Zhang, for the many laughs and for creating a fun and stimulating work environment. I would especially like to thank Dr. Christian Dansereau and Dr. Pierre Orban for their mentorship, and I was lucky to have them as my role models. I'm grateful for the entire SIMEXP team and I'm going to miss everyone terribly. I also thank Dr. Marie-Élyse Lafaille-Magnan for her solidarity and discussions. I also give thanks to my friends outside of academia for their tremendous support and encouragement. I want to thank Jacob Vogel for the long days and nights we spent working alongside each other and everything else in between. Lastly, I thank my family, especially my parents, my brother Brandon, and my grandparents, for their love and the many home-cooked meals that helped me get through every challenge, scientific or otherwise, as a better person.
\end{acknowledgements}



%----------------------------------------------------------------------------------------
%	DEDICATION
%----------------------------------------------------------------------------------------

\dedicatory{To my parents\ldots I'm sorry I got a job in Singapore so I probably won't come back for graduation.}

\cleardoublepage

%----------------------------------------------------------------------------------------
%	QUOTATION PAGE
%----------------------------------------------------------------------------------------

\vspace*{0.2\textheight}

\noindent\enquote{\itshape Reality is not what you perceive; it's what the methods and tools of science reveal.}\bigbreak

\hfill Neil deGrasse Tyson

%----------------------------------------------------------------------------------------
%	CONTRIBUTIONS PAGE
%----------------------------------------------------------------------------------------

\begin{contributions}
\addchaptertocentry{\contribs} % Add the declaration to the table of contents
As lead author of the studies described in chapters 3 through 5, I took the lead in executing the experiments. This included study design, data preprocessing, quality assessment, statistical analysis, visualization and interpretation of results, and writing of the manuscripts. I also received much appreciated and valuable help from my co-authors, outlined below.


%\noindent
%\textbf{Chapters 3 and 4}
\subsection*{Chapters 3 and 4}
\begin{itemize}
\setlength\itemsep{0em}
  \item Christian Dansereau: data preprocessing and quality control
  \item AmanPreet Badhwar: writing of the manuscript
  \item Pierre Orban: visualization of results
  \item Sylvie Belleville: review of the manuscript
  \item Howard Chertkow: advice for methods
  \item Alain Dagher: data collection
  \item Alexandru Hanganu: data collection
  \item Oury Monchi: data collection
  \item Pedro Rosa-Neto: data collection and curation
  \item Amir Shmuel: data collection
  \item Seqian Wang: data curation
  \item John Breitner: writing of the manuscript
  \item Pierre Bellec: study design, interpretation of results, and writing of the manuscript
\end{itemize}

\cleardoublepage

%\noindent
%\textbf{Chapter 5}
\subsection*{Chapter 5}
\begin{itemize}
\setlength\itemsep{0em}
\item Christian Dansereau: design of methods
\item Yasser Iturria-Medina: design of methods
\item Sebastian Urchs: design of methods
\item Pierre Orban: design of methods
\item John Breitner: review of manuscript
\item Pierre Bellec: study design, interpretation of results, and writing of manuscript
\end{itemize}


\end{contributions}

\cleardoublepage





%----------------------------------------------------------------------------------------
%	THESIS CONTENT - CHAPTERS
%----------------------------------------------------------------------------------------

\mainmatter % Begin numeric (1,2,3...) page numbering

\pagestyle{thesis} % Return the page headers back to the "thesis" style

% Include the chapters of the thesis as separate files from the Chapters folder
% Uncomment the lines as you write the chapters

% Chapter 1

\chapter{Introduction} % Main chapter title

\label{Chapter1} % Change X to a consecutive number; for referencing this chapter elsewhere, use \ref{ChapterX}

%----------------------------------------------------------------------------------------
%	SECTION 1
%----------------------------------------------------------------------------------------
\section{General context}

Alzheimer's disease (AD) is a progressive neurodegenerative disorder and the most common cause of dementia. As AD is an age-related disorder, more and more individuals will develop AD as the population ages. For instance, there are currently 53 million adults aged 65 and older in the United States in 2018. This segment of the American population is projected to increase to 88 million by 2050, followed by an expected doubling of the number of AD cases \citep{Association:2018eq}. Under current circumstances for diagnosis and treatment, the cost of medical and long-term care expenditures for American individuals who will develop AD in 2018 is projected to be \$47.1 trillion USD \citep{Association:2018eq}. AD clearly presents a public health crisis that requires imminent solutions.
% maybe change to Canadian??

Unfortunately, there is currently no disease-modifying drug for AD that can reverse or slow down the course of the disease. The failures for clinical trials of disease-modifying agents have been blamed on lack of efficacy, among other reasons \citep{Cummings:2014gn}. The lack of efficacy could be due to drugs being tested too late in the disease process when irreversible neurodegeneration has already occurred in clinical trial participants \citep{Aisen:2013fj}. The pathology of AD develops over decades prior to the emergence of clinical symptoms. Individuals who develop AD dementia typically experience an intermediate prodromal phase called mild cognitive impairment (MCI), where some cognitive deficits are apparent but not severe enough to impede daily function. There is therefore a push to target pre-symptomatic individuals with disease-modifying agents to try to prevent disease progression. A second reason for a lack of demonstrated efficacy could be heterogeneity within clinical populations. AD exists in several forms, and an individual's risk for developing AD depends on a variety of factors including, but not limited to, genetics, cardiovascular health, and brain reserve. This heterogeneity may explain why inclusion criteria for clinical trials have had low to moderate positive predictive value for diagnosing MCI subjects who would develop AD dementia \citep{Visser:2005iw}. The failure of clinical trials at the MCI stage in order to prevent dementia may be partially attributed to the incorrect inclusion of individuals who will not develop AD dementia. The development of a biomarker for early and precise diagnosis that would account for this heterogeneity could greatly improve patient selection for clinical trials and could identify high-risk individuals for earlier interventions. 

% An early biomarker would also be potentially valuable for clinical use. An early diagnosis prior to the emergence of symptoms may allow for early interventions. Although there are currently no pharmacological treatments that can slow down AD-related neurodegeneration, there are therapies (both pharmacologial and non-pharmacological) that may decrease the rate of cognitive decline in patients. Aside from medical benefits, an early diagnosis may have an important financial impact, where AD-related health costs could be significantly reduced with earlier detection \citep{Association:2018eq}.


%----------------------------------------------------------------------------------------
%	SECTION 2
%----------------------------------------------------------------------------------------

\section{Objectives}

A promising tool that is already widely used in clinical settings to aid in the diagnosis of AD is magnetic resonance imaging (MRI). The overall objective of this thesis was to explore heterogeneity within MRI-based biomarkers in order to discover a brain signature that is highly predictive of Alzheimer's dementia. Such a signature could include, for example, atrophy in the medial temporal lobes and specific nodes of the default mode network, dysconnectivity within those regions, and cognitive deficits. We focused on two MRI modalities, structural and functional MRI, and examined their potential to explain cognitive outcomes. Specific objectives from each chapter that presents original research are described below.

\subsection*{Chapter 3 objectives}
Chapter 3 of this thesis assessed the robustness of resting-state connectivity derived from functional MRI as an early biomarker for AD, in the face of heterogeneity from different image acquisition and diagnostic protocols. To this end, we combined multiple independent datasets of resting-state functional magnetic resonance images from MCI and cognitively normal subjects to test the consistency of functional connectivity differences between these two diagnostic groups.  

\subsection*{Chapter 4 objectives}
Chapter 4 was a companion paper to Chapter 3 and described resting-state network parcellations that are present in a large heterogeneous population of older adults with or without MCI. 

\subsection*{Chapter 5 objectives}
Chapter 5 of this thesis characterized the variability in brain atrophy patterns, derived from structural MRI, and characterized the heterogeneity to develop a signature that has high positive predictive value at predicting incipient AD dementia in MCI subjects. We achieved this by identifying subtypes of brain atrophy in AD and control subjects. We then applied a machine learning algorithm to classify AD and controls based on brain atrophy subtypes and cognitive test scores. This resulted in a signature that was common to AD patients but not represented in controls. We then transfered the predictive model to identify a subset of individuals who carried the signature and who would progress to AD dementia from a dataset of MCI subjects.




\include{Chapters/chap_2_review}
\include{Chapters/chap_3_rsfmri_mci}
\include{Chapters/chap_4_parcel}
\include{Chapters/chap_5_hps}
\include{Chapters/chap_6_discussion}

%----------------------------------------------------------------------------------------
%	BIBLIOGRAPHY
%----------------------------------------------------------------------------------------

\printbibliography[heading=bibintoc]

%----------------------------------------------------------------------------------------
%	THESIS CONTENT - APPENDICES
%----------------------------------------------------------------------------------------

%\appendix % Cue to tell LaTeX that the following "chapters" are Appendices

\begin{appendices}

% Include the appendices of the thesis as separate files from the Appendices folder
% Uncomment the lines as you write the Appendices

\include{Appendices/appendix_1}
\include{Appendices/appendix_2}
\include{Appendices/appendix_3}

\end{appendices}


%----------------------------------------------------------------------------------------

\end{document}
